\documentclass[10pt]{beamer}
\usepackage[utf8x]{inputenc}
\usepackage{hyperref}
\usepackage{fontawesome}
\usepackage{graphicx}
\usepackage[english,ngerman]{babel}
\graphicspath{ {./images/} }
% ------------------------------------------------------------------------------
% Use the beautiful metropolis beamer template
% ------------------------------------------------------------------------------
\usepackage[T1]{fontenc}
\usepackage{fontawesome}
\usepackage{FiraSans} 
\mode<presentation>
{
  \usetheme[progressbar=foot,numbering=fraction,background=light]{metropolis} 
  \usecolortheme{crane} % or try albatross, beaver, crane, ...
  \usefonttheme{structurebold}  % or try serif, structurebold, ...
  \setbeamertemplate{navigation symbols}{}
  \setbeamertemplate{caption}[numbered]
  %\setbeamertemplate{frame footer}{My custom footer}
} 

% ------------------------------------------------------------------------------
% beamer doesn't have texttt defined, but I usually want it anyway
% ------------------------------------------------------------------------------
\let\textttorig\texttt
\renewcommand<>{\texttt}[1]{%
  \only#2{\textttorig{#1}}%
}

% ------------------------------------------------------------------------------
% minted
% ------------------------------------------------------------------------------
\usepackage{minted}


% ------------------------------------------------------------------------------
% tcolorbox / tcblisting
% ------------------------------------------------------------------------------
\usepackage{xcolor}
\definecolor{codecolor}{HTML}{FFC300}

\usepackage{tcolorbox}
\tcbuselibrary{most,listingsutf8,minted}

\tcbset{tcbox width=auto,left=1mm,top=1mm,bottom=1mm,
right=1mm,boxsep=1mm,middle=1pt}

\newtcblisting{myr}[1]{colback=codecolor!5,colframe=codecolor!80!black,listing only, 
minted options={numbers=left, style=tcblatex,fontsize=\tiny,breaklines,autogobble,linenos,numbersep=3mm},
left=5mm,enhanced,
title=#1, fonttitle=\bfseries,
listing engine=minted,minted language=r}


% ------------------------------------------------------------------------------
% Listings
% ------------------------------------------------------------------------------
\definecolor{mygreen}{HTML}{37980D}
\definecolor{myblue}{HTML}{0D089F}
\definecolor{myred}{HTML}{98290D}

\usepackage{listings}

% the following is optional to configure custom highlighting
\lstdefinelanguage{XML}
{
  morestring=[b]",
  morecomment=[s]{<!--}{-->},
  morestring=[s]{>}{<},
  morekeywords={ref,xmlns,version,type,canonicalRef,metr,real,target}% list your attributes here
}

\lstdefinestyle{myxml}{
language=XML,
showspaces=false,
showtabs=false,
basicstyle=\ttfamily,
columns=fullflexible,
breaklines=true,
showstringspaces=false,
breakatwhitespace=true,
escapeinside={(*@}{@*)},
basicstyle=\color{mygreen}\ttfamily,%\footnotesize,
stringstyle=\color{myred},
commentstyle=\color{myblue}\upshape,
keywordstyle=\color{myblue}\bfseries,
}

% ------------------------------------------------------------------------------
% The Document
% ------------------------------------------------------------------------------
\title{ECS in Game Development}
\author{Jorge Pinto Sousa (He/Him), Luiz Jardim (Viceroy/Viceroyer)}

\institute{Critical Techworks}
\date{July 2021}
\usemintedstyle[yaml]{}
\newcommand{\cpp}{C++}
\newcommand\myheading[1]{%
  \par\bigskip
  {\Large\bfseries#1}\par\smallskip}



\begin{document}

\begin{frame}
\titlepage
\end{frame}

\section{Entity–component–system (ECS)}
\begin{frame}[fragile]{Entity–component–system (ECS)}

\begin{itemize}
    \item   
        Entity-Component-System is a software architecture pattern frequently used in game development, often together with a data oriented design. 
    \item
        It addresses some of the problems with object orientation while promoting code reusability, extendability, maintanability and paralle... para... parallelizability (is this a word?).
    \item 
        One of it's greatest features is easily modifying behaviour at runtime.
\end{itemize}


\end{frame}

\begin{frame}[ragile]{Entity–component–system (ECS)}
\begin{center}
\myheading{Composition over inheritance}
\end{center}
\end{frame}

\begin{frame}[fragile]{Entity–component–system (ECS)}
\begin{itemize}
    \item   
        \textbf{Entities} are unique "things" that are assigned groups of \textbf{Components}, which are then processed using \textbf{Systems}.
\end{itemize}
\end{frame}

\begin{frame}[fragile]{Entity–component–system (ECS)}
\myheading{What does this mean?}
\begin{itemize}
    \item Imagine that you have an \mintinline{bash}{Entity} (a Villain for example), that has a \mintinline{bash}{Position} and a \mintinline{bash}{Velocity} Component, and a NPC which has a \mintinline{bash}{Position} and \mintinline{bash}{UI} component.
    \item Then you can have a \mintinline{bash}{Movement} system that will run on all entities that have a \mintinline{bash}{Position} and \mintinline{bash}{Velocity} component, for example.
\end{itemize}
\end{frame}
\end{document}f
